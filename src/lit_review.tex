\section{Literature Review}

Object detection from images involves two tasks: object classification and object locating. 
Both of these tasks involve the extraction of meaningful features from an image. 
In modern CNN approaches to object detection, the feature extraction tends to be layers within the models architecture \cite{reddi2024mlsystems}, while traditional computer vision approaches required engineered feature extraction methods.

One such feature extraction method is the Histogram of Oriented Gradients (HOG) method which captures local gradient structures from cells within an image \cite{HOGpaper}, in fact in their paper the authors show "essentially perfect" classification of pedestrians in a MIT prediction test set.
Other popular feature extraction methods at the time included the Scale-Invariant Feature Transform (SIFT) method which captures local invariant features from an image \cite{lowe2004distinctive}.

% Classification
For the classification and detection tasks, different approaches to the learned Support Vector Machines (SVM) have been used.
In \cite{papageorgiou2000trainable}, the authors detail a pedestrian detector based on a polynomial SVM using rectified Haar wavelets.
A human limb classifier using a SVM and 1st and 2nd Gaussian features is developed in \cite{ronfard2002learning}.

Other previously used learning based approaches include works such as \cite{1238422}, where the authors build a person detector using AdaBoost and Haar-like features.