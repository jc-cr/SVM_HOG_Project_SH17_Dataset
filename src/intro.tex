\section{Introduction}

Object detection is a core task in computer vision research and development. 
Modern deep learning approaches like convolutional neural networks (CNN) have been the standard approach to the object detection for RGB images ever since their capabilities for training with GPU hardware was demonstrated in 2012 \cite{alexnet}.
Since then architectures such as YOLO \cite{yolo} and Faster R-CNN \cite{ren2016fasterrcnnrealtimeobject} have been developed to improve the speed and accuracy of object detection tasks.

In this  report my goal is to evaluate the performance of a more traditional object detection approach, the Histogram of Oriented Gradients (HOG) combined with a linear Support Vector Machine (SVM) classifier, on a recent dataset, the Safe Human dataset, a workplace safety dataset with 17 different objects \cite{ahmad2024sh17datasethumansafety}.
The linear SVM approach is chosen as that is what is used in the original HOG paper \cite{HOGpaper}, in addition because linear SVM approaches are considered to be fast at inference time while requiring less computational resources for training compared to deep learning approaches such as a CNN.