\section{Problem Description}

The SH17 dataset contains 8,099 annotated images with 75,994 instances across 17 classes. 
A person detection model is then implemented using HOG features extraction method and a classified with a linear SVM model.
The performance of the model is evaluated using COCO benchmarks and compared to benchmarks obtained in the the SH-17 paper which uses YOLO9 and YOLO10 architectures.