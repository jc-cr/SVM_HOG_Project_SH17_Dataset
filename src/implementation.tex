\section{Implementation}


For the evaluation pipeline, the sliding window approach is again used but incorporates a image scaling pyramid to handle size variations of the target objects similar to that used in (plvs2023hogdetection).
In the sliding window the HOG feature extractor method is again used with the same parameters as in the training step.

The  trained SVM model provides confidence scores for each window using decision_function(), which returns the signed distance from the separating hyperplane, we then keep those above a certain threshold as detections.

Without post-processing the above method generates many "positive" predictions as seen in Figure X. 
This is problematic as each prediction is considered to be person thus negatively impacting the COCO metric calculations with the false positives. 
To account for this, post processing methods are added to standardizes the detection aspect ratios, merge overlapping boxes, and the do a non-maximum suppression to keep only the highest scoring boxes.
It's worth noting that since the model used a scaler in it's pipeline, features with scores above 0.0 are considered within the "person" classification, while those below  0.0 are negative.

After 2 post processing steps, the prediction boxes are consolidated and merged resulting in results such as Figure X.